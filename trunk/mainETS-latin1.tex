\documentclass [memoire, letterpaper, oneside, fleqn,12pt]{thETS-latin1}
% pour une ma�trise rajouter l'option memoire � la classe (defaut:these)
% pour activer les liens hypertextes, ajouter l'option hypertext

\listfiles

%------------------------------------------------------------------------------------------------------------------------------------------

% � d�commanter et modifier si n�cessaire dans le cas d'une maitrise
\genie{Logiciel} %\genie{�lectrique}

\title{titre de la th�se}

\author{Pr�nom NOM}
\authorcopyright{Pr�nom Nom}

\datesoutenance{\today}

\datedepot{\today}

\directeur{M. }{Pr�nom Nom}{d�partement et institution}

\codirecteur{Mme.}{Pr�nom Nom}{d�partement et institution}

\president{M.}{Pr�nom Nom}{d�partement et institution}

\jury{Mme.}{Pr�nom Nom}{d�partement et institution}{}

\examinexterne{M.}{Pr�nom Nom}{d�partement et institution}{}

%------------------------------------------------------------------------------------------------------------------------------------------

\begin{document}

\pagenumbering{roman}

%%%%%%%%%%%%%%%%%%%%%%%%%%%%%%%%%%%%%%%%%%%%%%%%%%%
% PAGE TITRE
%%%%%%%%%%%%%%%%%%%%%%%%%%%%%%%%%%%%%%%%%%%%%%%%%%%
\maketitle

%%%%%%%%%%%%%%%%%%%%%%%%%%%%%%%%%%%%%%%%%%%%%%%%%%%
% PR�SENTATION JURY
%%%%%%%%%%%%%%%%%%%%%%%%%%%%%%%%%%%%%%%%%%%%%%%%%%%
\presentjury

%%%%%%%%%%%%%%%%%%%%%%%%%%%%%%%%%%%%%%%%%%%%%%%%%%%
% AVANT PROPOS
%%%%%%%%%%%%%%%%%%%%%%%%%%%%%%%%%%%%%%%%%%%%%%%%%%%
\begin{avantpropos}

\end{avantpropos}

%%%%%%%%%%%%%%%%%%%%%%%%%%%%%%%%%%%%%%%%%%%%%%%%%%%
% REMERCIEMENTS
%%%%%%%%%%%%%%%%%%%%%%%%%%%%%%%%%%%%%%%%%%%%%%%%%%%
\begin{remerciements}

\end{remerciements}

%%%%%%%%%%%%%%%%%%%%%%%%%%%%%%%%%%%%%%%%%%%%%%%%%%%
% SOMMAIRE
%%%%%%%%%%%%%%%%%%%%%%%%%%%%%%%%%%%%%%%%%%%%%%%%%%%
\motscles{mot-cl�1, mot-cl�2}
\begin{sommaire}
Ceci est le r�sum�.
\end{sommaire} 


%%%%%%%%%%%%%%%%%%%%%%%%%%%%%%%%%%%%%%%%%%%%%%%%%%%
% ABSTRACT
%%%%%%%%%%%%%%%%%%%%%%%%%%%%%%%%%%%%%%%%%%%%%%%%%%%
\keywords{keyword1, keyword2}
\begin{abstract}{title in english}
Here is the abstract.
\end{abstract}


%%%%%%%%%%%%%%%%%%%%%%%%%%%%%%%%%%%%%%%%%%%%%%%%%%%
% TABLE DES MATI�RES
%%%%%%%%%%%%%%%%%%%%%%%%%%%%%%%%%%%%%%%%%%%%%%%%%%%
\tableofcontents

%%%%%%%%%%%%%%%%%%%%%%%%%%%%%%%%%%%%%%%%%%%%%%%%%%%
% LISTE DES TABLEAUX
%%%%%%%%%%%%%%%%%%%%%%%%%%%%%%%%%%%%%%%%%%%%%%%%%%%
\listoftables

%%%%%%%%%%%%%%%%%%%%%%%%%%%%%%%%%%%%%%%%%%%%%%%%%%%
% LISTE DES FIGURES
%%%%%%%%%%%%%%%%%%%%%%%%%%%%%%%%%%%%%%%%%%%%%%%%%%%
\listoffigures

%%%%%%%%%%%%%%%%%%%%%%%%%%%%%%%%%%%%%%%%%%%%%%%%%%%
% LISTE DES EXTRAITS DE CODE
%%%%%%%%%%%%%%%%%%%%%%%%%%%%%%%%%%%%%%%%%%%%%%%%%%%
\lstlistoflistings

%%%%%%%%%%%%%%%%%%%%%%%%%%%%%%%%%%%%%%%%%%%%%%%%%%%
% LISTE DES ABR�VIATIONS
%%%%%%%%%%%%%%%%%%%%%%%%%%%%%%%%%%%%%%%%%%%%%%%%%%%
\begin{listofabbr}[3cm]
\item[SDA] Signification De l'Abr�viation
\end{listofabbr}

%%%%%%%%%%%%%%%%%%%%%%%%%%%%%%%%%%%%%%%%%%%%%%%%%%%
% LISTE DES SYMBOLES
%%%%%%%%%%%%%%%%%%%%%%%%%%%%%%%%%%%%%%%%%%%%%%%%%%%
\begin{listofsymbols}[3cm]
\item[symbol] signification
\end{listofsymbols}

%%%%%%%%%%%%%%%%%%%%%%%%%%%%%%%%%%%%%%%%%%%%%%%%%%%
% CORPS DE LA TH�SE
%%%%%%%%%%%%%%%%%%%%%%%%%%%%%%%%%%%%%%%%%%%%%%%%%%%
\newpage
\pagenumbering{arabic}

\reversemarginpar % pour que les marginpar s'amenent a GAUCHE du doc.

\begin{introduction}

\end{introduction}

\chapter{Chapitre}
\section{Section}
\subsection{Sous-Section}
\subsubsection{Sous-Sous-Section}

\subsubsection{Code}
Ceci est un extrait de code~:
\begin{lstlisting}
[language=Matlab,caption=Initialisation,label=lst:init]
for i=1:N
  tab(i)=rand(1,1);
end
\end{lstlisting}

\subsubsection{Tableau}
Ceci est un tableau~:
\begin{table}[ht]
    \begin{tabular}{|c|c|c|c|}\hline
      bl� & bl� & bl� & bl� \\\hline
      bl� & bl� & bl� & bl� \\\hline
    \end{tabular}
  \caption{Petit tableau}
  \label{tab:UnTableau}
\end{table}

\subsubsection{Figure}
Ceci est une figure~:
\begin{figure}[h]
  \fbox{ % cette commande est n�c�ssaire pour encadrer les figures
    \centering
    \includegraphics[width=.8\textwidth]{logoets.jpg}
  }
  \caption{Logo de l'�cole de Technologie Sup�rieure. Tir� de \cite{ETS2010}.}
  \label{fig:logoets}
\end{figure}

%\include{chapitre1}
%\include{chapitre2}
%\include{chapitre3}
%\include{chapitre4}
%\include{chapitre5}
%\include{chapitre6}
%\include{chapitre7}
%\include{chapitre8}

\begin{conclusion}

\end{conclusion}

\begin{remerciements}

\end{remerciements}

%%%%%%%%%%%%%%%%%%%%%%%%%%%%%%%%%%%%%%%%%%%%%%%%%%%
%  ANNEXE:
%%%%%%%%%%%%%%%%%%%%%%%%%%%%%%%%%%%%%%%%%%%%%%%%%%%
\appendix
\multiannexe % si on a plus d'une annexe
%\include{annexe1}
\chapter{Titre de l'annexe}
S'il y lieu

%%%%%%%%%%%%%%%%%%%%%%%%%%%%%%%%%%%%%%%%%%%%%%%%%%%
% BIBLIOGRAPHIE
%%%%%%%%%%%%%%%%%%%%%%%%%%%%%%%%%%%%%%%%%%%%%%%%%%%
\bibliographystyle{plain-fr}
%\bibliographystyle{plainnat-fr} %With \usepackage[authoryear]{natbib}
\renewcommand{\bibname}{R�F�RENCES BIBLIOGRAPHIQUES}
\addcontentsline{toc}{chapter}{R�F�RENCES BIBLIOGRAPHIQUES}
%\bibliography{nom_fichier_bib} % � d�commander et indiquer la liste des fichiers bib

%------------------------------------------------------------------------------------------------------------------------------------------

\end{document}
